% --------------------------------------------------------------
% This is all preamble stuff that you don't have to worry about.
% Head down to where it says "Start here"
% --------------------------------------------------------------
 
\documentclass[12pt]{article}
 
\usepackage[margin=1in]{geometry} 
\usepackage{amsmath,amsthm,amssymb}
\usepackage[margin=1in]{geometry} 
\usepackage{amsmath,amsthm,amssymb}
\usepackage[spanish]{babel} %Castellanización
\usepackage[T1]{fontenc} %escribe lo del teclado
\usepackage[utf8]{inputenc} %Reconoce algunos símbolos
\usepackage{lmodern} %optimiza algunas fuentes
\usepackage{graphicx}
\graphicspath{ {images/} }
\usepackage{hyperref} % Uso de links
\usepackage[most]{tcolorbox}
\usepackage{listings}
 
\newcommand{\N}{\mathbb{N}}
\newcommand{\Z}{\mathbb{Z}}
 
\newenvironment{theorem}[2][Theorem]{\begin{trivlist}
\item[\hskip \labelsep {\bfseries #1}\hskip \labelsep {\bfseries #2.}]}{\end{trivlist}}
\newenvironment{lemma}[2][Lemma]{\begin{trivlist}
\item[\hskip \labelsep {\bfseries #1}\hskip \labelsep {\bfseries #2.}]}{\end{trivlist}}
\newenvironment{exercise}[2][Exercise]{\begin{trivlist}
\item[\hskip \labelsep {\bfseries #1}\hskip \labelsep {\bfseries #2.}]}{\end{trivlist}}
\newenvironment{problem}[2][Problem]{\begin{trivlist}
\item[\hskip \labelsep {\bfseries #1}\hskip \labelsep {\bfseries #2.}]}{\end{trivlist}}
\newenvironment{question}[2][Pregunta]{\begin{trivlist}
\item[\hskip \labelsep {\bfseries #1}\hskip \labelsep {\bfseries #2.}]}{\end{trivlist}}
\newenvironment{corollary}[2][Corollary]{\begin{trivlist}
\item[\hskip \labelsep {\bfseries #1}\hskip \labelsep {\bfseries #2.}]}{\end{trivlist}}

\newenvironment{solution}{\begin{proof}[Solución]}{\end{proof}} %It just adds Proof in italics at the beginning of the text given as argument and a white square (Q.E.D. symbol, also known as a tombstone) at the end of it.
\renewcommand{\qedsymbol}{} % To hide the Q.E.D. symbol altogether, redefine it to be blank:






%--------------------PARAMETRIZA APARIENCIA DE CODIGO FUENTE ------------------------------------
\usepackage{listings}
\usepackage{xcolor}
%New colors defined below
\definecolor{codegreen}{rgb}{0,0.6,0}
\definecolor{codegray}{rgb}{0.5,0.5,0.5}
\definecolor{codepurple}{rgb}{0.58,0,0.82}
\definecolor{backcolour}{rgb}{0.95,0.95,0.92}

%Code listing style named "mystyle"
\lstdefinestyle{mystyle}{
  backgroundcolor=\color{backcolour},   commentstyle=\color{codegreen},
  keywordstyle=\color{magenta},
  numberstyle=\tiny\color{codegray},
  stringstyle=\color{codepurple},
  basicstyle=\ttfamily\footnotesize,
  breakatwhitespace=true,         
  breaklines=true,                 
  captionpos=b,                    
  keepspaces=true,                 
  numbers=left,                    
  numbersep=5pt,                  
  showspaces=false,                
  showstringspaces=false,
  showtabs=false,                  
  tabsize=2
}

\lstset{style=mystyle}


%------------------------------------------------------------------------------------------------
 
 
\begin{document}
 
% --------------------------------------------------------------
%                         Start here
% --------------------------------------------------------------
 
\title{TITULO DEL TRABAJO A ENTREGAR}
\author{Juan Manuel Agudelo Aguirre\\ %replace with your name
Docente: Cesar Manuel Castillo Rodriguez\\
Universidad Tecnológica de Pereira}
\date{06 de Agosto de 2024}

\maketitle



\begin{center}
\section*{subtitulos \textit{cursivas} y \textit{cursivas}}    
\end{center}

 
\begin{question}{1}
primera pregunta\\

Su programa debe mostrar un mensaje de error apropiado si el primer valor ingresado por el usuario es 0.


\begin{tcolorbox}
caja de texto.
\end{tcolorbox}

\end{question}

\begin{lstlisting}[language=Python]
modifique aqui su codigo
\end{lstlisting}
%-------------------------------------------------------------
\begin{question}{9}
Modifica tablas
\begin{center}
    \begin{tabular}{|c|c|}
    \hline
         \textbf{Producto} & \textbf{Precio} \\
        \hline
         Brocha & 500  \\
         \hline
         Espatula & 2000 \\
         \hline
         Pala & 54000 \\
         \hline
         Carretilla & 300000 \\
         \hline
         Casco & 55000 \\
         \hline
         Soldador & 230000 \\
         \hline
         Alicate & 10000 \\
         \hline
         Destornillador & 3000 \\
         \hline
         Maza & 60000 \\
         \hline
         Nivel & 24000 \\
         \hline
         Flexometro & 76000 \\
         \hline
         Hacha & 32000 \\
         \hline
         Pico & 74000 \\
         \hline
         Rastrillos & 56000\\
         \hline
         
    \end{tabular}
\end{center}
\end{question}


\begin{center}
    \begin{tabular}{|c|c|c|}
    \hline
         \textbf{Banda} & \textbf{Miembros} \\
        \hline
         Queen & Freddie Mercury, Brian May, Roger Taylor y John Deacon  \\
         \hline
         The Doors & Jim Morrison, Robby Krieger, Ray Manzarek y John Densmore \\
         \hline
         Green Day & Billie Joe Armstrong, Mike Dirnt  y Tré Cool  \\
         \hline
         Rolling Stones & Mick Jagger, Keith Richards, Charlie Watts y Ron Wood  \\
         \hline
         Led zeppelin & Robert Plant, Jimmy Page,  John Paul Jones y John Bonham.  \\
         \hline
         Black Sabbath & Ozzy Osbourne, Tony Iommi, Bill Ward y Geezer Butler \\
         \hline
         Soundgarden & Chris Cornell, Kim Thayil, Matt Cameron y Ben Shepherd. \\
         \hline
         Alice in Chains &  Layne Staley, Jerry Cantrell, Sean Kinney y Mike Starr \\
         \hline
         
    \end{tabular}
\end{center}

Modifique sus listas

\begin{itemize}
\item primer elemento de la lista
\item segundo elemento de la lista
\end{itemize}
\end{document}